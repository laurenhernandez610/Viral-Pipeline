\documentclass{article}
\title{Analysis of Hospitalization, Positive Covid-19 Viral Tests and Recovered Individuals}
\author{Lauren H, Ethan, Jacob, Wolf\\}

\date{\today}
\usepackage{pdfpages}
\usepackage{subfigure}
\usepackage{graphicx} 
\usepackage{placeins}
\begin{document}

\maketitle
The scope of this analysis concentrates on Texas and specific trauma service areas within Texas.


\section{Daily Hospitalizations in Texas}

The data sets include new hospitalizations statewide and new hospitalizations within specific regions of Texas that are referred to as trauma service areas. In 1992, Texas was subdivided into twenty-two trauma service areas in order to establish a trauma response system that would be unique and best fitting to each region, with the intent to minimize casualty due to trauma, natural disaster or bioterrorist attack. Further, each trauma service area is assigned an initial (A-V) and is informally referred to by its largest, or most well-known, city within its boundaries. 

\vspace{2mm}

Figure 2 displays data for Texas overall and for the four most severely affected trauma service areas in Texas regarding new lab confirmed Covid-19 hospitalizations. The trauma service areas in Figure 2 are: TSA E, which represents the Dallas/ Fort Worth area; TSA P, which represents the San Antonio area; TSA Q, which represents the Houston area; TSA V, which represents the Lower Rio Grande Valley.  Figure 3 displays data for seven trauma service areas in Texas that are currently reporting over one hundred new lab confirmed Covid-19 hospitalizations per day. The trauma service areas in Figure 3 are: TSA F, which represents the Paris area; TSA G, which represents the Longview/Tyler area; TSA I, which represents the El Paso area; TSA O, which represents the Austin area; TSA R, which represents the Galveston area; TSA T, which represents the  Laredo area; TSA U, which represets the Corpus Christi area. 

\vspace{1cm}
\begin{figure}[!htbp]
	\centering
	\includegraphics[scale=0.4]{RAC map}
	\caption{\footnotesize{Trauma Service Areas within Texas and counties associated.} }
	\label{fig:1}
\end{figure}		
	\FloatBarrier

The two graphs in Figure 2 show the progression of new, or daily, lab confirmed Covid-19 hospitalizations in Texas from April 12, 2020 to August 19, 2020. The data shows that from April 12, 2020 to approximately June 1, 2020, new hospitalizations in the state of Texas were stable and below 2,000 new hospitalizations per day statewide; similarly, the data shows that each trauma service area listed was stable and below 1,000 new hospitalizations per day. However, on June 1, 2020 each dataset grows exponentially and plateaus to a local maximum around approximately July 20, 2020. 

\vspace{2mm}

On April, 30, 2020, Governor Greg Abbott's statewide stay at home order expires and Phase I of reopening Texas begins the following day, May 1, 2020. During Phase I of reopening, select businesses are permitted to open with a 25 percent occupancy. During Phase II of reopening, on May 18, 2020, select businesses are permitted to open with a fifty percent occupancy and bars are allowed to open at a 25 percent ocupancy. On June 3, 2020, Phase III of reopening takes place and all businesses are allowed to open at 50 percent occupancy. On June, 12, 2020, restaurants can open at 75 percent occupancy. Around middle to late June, the mentality towards the reopening of Texas begins to shift and on June 26, 2020, Governor Greg Abbott orders the closing of all bars and restaurant occupancy is reduced to 50 percent. Concludingly, on July 2, 2020, Governor Greg Abbott issues a mask mandate in public spaces. 

\vspace{2mm}

Daily lab confirmed Covid-19 hospitalizations in Texas have changed over time in response to each phase of reopening initiated by Governor Greg Abbott. The data shows that while Texas was operating under a stay at home order, daily hospializations were stable throughout Texas. Approximately two weeks after Phase II of reopening Texas, roughly June 1, 2020, daily lab confirmed Covid-19 hospitalizations begin to rise at an increasing rate for appproximately six weeks. The most dramatic increase in daily lab confirmed Covid-19 hospitalizations are seen in TSA P and TSA V, corresponding to the San Antonio area and the Lower Rio Grande Valley area. At the end of June, June 26, 2020, Governor Greg Abbott reverses the reopening of Texas and specifically orders the closing of bars and reduces the occupancy of restaurants down to fifty percent. The data shows that approximately two and a half weeks after Governor Greg Abbott selectively reverses the reopening of Texas, the reportings of new lab confirmed Covid-19 hospitalizations reach a peak and then begin to steadily decline. At the end of July, each dataset shows a fluctuating decline in reportings that stabalizes, with little to no fluctuation, around August 1, 2020. This fluctuation in reportings is due to the Texas Department of State Health Services adjusting their methods of data reporting to comply with new federal requirements. The data shows that as the Texas Department of State Health Service is adjusting their data recording of new lab confirmed Covid-19 hospitalizations, there is a subtle to dramatic decline in new reportings shown by each dataset. The most probable reason for this sudden decline and stabilization of new lab confirmed Covid-19 hospitalizations is that some hospitalizations were unaccounted for, or rather, lost during the transition of data recording to comply with new federal regulations. Further, the data shows that a transition in data recording resulted in a brief period of inaccurate and incomplete data representation regarding new lab confirmed Covid-19 hospitalizations in Texas and select trauma service areas.

\vspace{2mm}

In conclusion, 


\begin{figure}[!htbp]
	\begin{center}
		\includegraphics[scale=0.45]{Final Draft A}
		\vspace{1mm}
		\includegraphics[scale=0.45]{Final Draft B}
		\caption{Daily Hospializations in Texas and four Trauma Service Areas within Texas that are reporting highest daily hospitalization due to Covid-19. The top figure displays the data on a linear scale and the bottom figure displays the data on a logarithmic scale. }
		\label{fig:2}
	\end{center}
\end{figure}

\FloatBarrier

\begin{figure}[!htbp]
	\begin{center}
		\includegraphics[scale=0.45]{Final Draft C}
		\vspace{1mm}
		\includegraphics[scale=0.45]{Final Draft D}
		\caption{Daily Hospializations in seven Trauma Service Areas within Texas that are reporting over one hundred daily lab confirmed hospitalizations due to Covid-19. The top figure displays the data on a linear scale and the bottom figure displays the data on a logarithmic scale. }
		\label{fig:3}
	\end{center}
\end{figure}

\FloatBarrier

\section{Viral Tests and Recovered Individuals in Texas}

The data here compares new positive Covid-19 viral tests to the daily count of new individuals who previously received a positive Covid-19 viral test and are now cosidered to be recovered. The data in Figure 4 and Figure 5 concerns individuals in Texas only. 

\vspace{2mm}

Figure 4 shows the two datasets, new positive Covid-19 viral tests and new recovered individuals, over time as they are being reported. The data in Figure 4 has been modified to show the trend over a seven day moving average period. The datasets in Figure 4 exhibit similar trends, since the dataset of new recovered individuals is a subset of new positive viral tests. 
The data in Figure 5 has been modified to show the trend over a seven day moving average period and adjusted to show a fourteen day lag. The fourteen day lag was applied by means of transplanting the dataset of new recovered individuals back in time by fourteen days in order to superimpose the two datasets for a visual comparison of similarities and differences.

\vspace{2mm}

Further, Figure 4  shows that the two datasets resemble one another and Figure 5 shows that when a fourteen day lag is applied to the two datasets, that the two datasets nearly overlap exactly. It can be inferred that on average, when an individual tests positive for a Covid-19 viral test in Texas, that approximately fourteen days later, the same individual will be considered to be recovered from Covid-19. 

In addition, Figure 4 and Figure 5 show the dramatic increase in positive Covid-19 viral tests over time in Texas, as well as the individuals who have recovered from a positive Covid-19 viral test over time. The data shows that from April 1, 2020 to about June 5, 2020, daily positive Covid-19 viral tests increase at a mild to moderate rate, hovering just below 1,750 new positive Covid-19 viral tests per day. The data shows that from approximately June 5, 2020 to approximately July 15, 2020, the daily new positive Covid-19 viral test count increases exponentilly to a local maximum of about 10,000 new positive Covid-19 viral tests per day in Texas. Further, the data shows that from approximately July 15, 2020 to approximately July 25, 2020, the daily count for new positive Covid-19 viral tests steadily decreases; however, from approximately July 25, 2020 to roughly August 17, 2020, the daily count for new positive Covid-19 viral tests rises and falls within brief windows of time. Noteworthy, the decline in new positive Covid-19 viral tests in Texas begins to decline roughly ten days after Governor Greg Abbott reverses the reopening of Texas by initiating the closing of bars and reducing restaurant occupancy down from seventy-five percent to fifty percent. 


\begin{figure}[!htbp]
	\begin{center}
		\includegraphics[scale=0.45]{Final Draft 10.pdf}
		\vspace{1mm}
		\includegraphics[scale=0.45]{Final Draft 11.pdf}
		\caption{Daily positive viral tests for Covid-19 in Texas and daily new recovered individuals (who previously received a positive Covid-19 viral test) in Texas adjusted by a seven day moving average. The top figure displays the data on a linear scale and the bottom figure displays the data on a logarithmic scale. }
		\label{fig:4}
	\end{center}
\end{figure}

\FloatBarrier

\begin{figure}[!htbp]
	\begin{center}
		\includegraphics[scale=0.45]{Final Draft 14.pdf}
		\vspace{1mm}
		\includegraphics[scale=0.45]{Final Draft 15.pdf}
		\caption{Daily positive viral tests for Covid-19 in Texas and daily new recovered individuals (who previously received a positive Covid-19 viral test) in Texas adjusted by a fourteen day lag and a seven day moving average. The top figure displays the data on a linear scale and the bottom figure displays the data on a logarithmic scale. }
		\label{fig:5}
	\end{center}
\end{figure}

\FloatBarrier

\begin{figure}[!htbp]
	\begin{center}
		\includegraphics[scale=0.45]{Final Draft G}
		\vspace{1mm}
		\includegraphics[scale=0.45]{Final Draft H}
		\caption{Correlation between daily positive viral tests for COvid-19 om Texas and daily recovered individuals (who previously received a positive Covid-19 viral test) in Texas versus lag time. The top figure shows the correlaton versus lag time for the entirety of the two datasets; the bottom figure shows the peak correlation versus lag time between the two datasets, where the highest correlation is at a lag time of twelve days. }
		\label{fig:6}
	\end{center}
\end{figure}

\FloatBarrier

\section{Los Angeles County, California}

\begin{figure}[!htbp]
	\begin{center}
		\includegraphics[scale=0.77]{LA 1}
		\vspace{1mm}
		\includegraphics[scale=0.77]{LA 1.5 }
		\caption{Daily lab-confirmed positive Covid-19 cases in Los Angeles County and Daily Covid-19 related deaths in Los Angeles County.The top figure displays the data on a linear scale and the bottom figure displays the data on a logarithmic scale.  }
		\label{fig:7}
	\end{center}
\end{figure}

\FloatBarrier

\begin{figure}[!htbp]
	\begin{center}
		\includegraphics[scale=0.77]{LA 2}
		\vspace{1mm}
		\includegraphics[scale=0.77]{LA 2.5 }
		\caption{Cumulative lab-confirmed positive Covid-19 cases in Los Angeles County and cumulative Covid-19 related deaths in Los Angeles County.The top figure displays the data on a linear scale and the bottom figure displays the data on a logarithmic scale. }
		\label{fig:8}
	\end{center}
\end{figure}

\FloatBarrier

\begin{figure}[!htbp]
	\begin{center}
		\includegraphics[scale=0.7]{LA 3}
		\vspace{1mm}
		\includegraphics[scale=0.7]{LA 4}
		\caption{ The top figure shows a scatterplot for cumulative lab-confirmed Covid-19 cases in Los Angeles County versus cumulative Covid-19 related deaths in Los Angeles County. The bottom figure shows the correlation between cumulative lab-confirmed Covid-19 cases and Covid-19 related deaths in Los Angeles County versus lag time, where the highest correlation between the two data sets is at day zero. (i.e. the datastets display the highest correlation when there is no lag time applied)}
		\label{fig:9}
	\end{center}
\end{figure}

\FloatBarrier

\begin{figure}[!htbp]
	\begin{center}
		\includegraphics[scale=0.67]{LA 5}
		\vspace{1mm}
		\includegraphics[scale=0.67]{LA 5.5 }
		\caption{ Daily lab-confirmed Covid-19 positive cases plotted alongside individuals currently admitted to the hospital (including lab-cofirmed Covid-19 positive individuals and symptomatic, Covid-19 suspected individuals who are awaiting the return of Covid-19 lab test results) and individuals currently in the ICU (including lab-cofirmed Covid-19 positive individuals and symptomatic, Covid-19 suspected individuals who are awaiting the return of Covid-19 lab test results). The top figure displays the data on a linear scale and the bottom figure displays the data on a logarithmic scale. Both figures display datasets representing Los Angeles County, California. }
		\label{fig:10}
	\end{center}
\end{figure}

\FloatBarrier


\section{San Bernardino County, California}

\begin{figure}[!htbp]
	\begin{center}
		\includegraphics[scale=0.77]{SB 1}
		\vspace{1mm}
		\includegraphics[scale=0.77]{SB 1.5 }
		\caption{Daily lab-confirmed positive Covid-19 cases in San Bernardino County and Daily Covid-19 related deaths in San Bernardino County.The top figure displays the data on a linear scale and the bottom figure displays the data on a logarithmic scale.  }
		\label{fig:11}
	\end{center}
\end{figure}

\FloatBarrier

\begin{figure}[!htbp]
	\begin{center}
		\includegraphics[scale=0.77]{SB 2}
		\vspace{1mm}
		\includegraphics[scale=0.77]{SB 2.5 }
		\caption{Cumulative lab-confirmed positive Covid-19 cases in San Bernardino County and cumulative Covid-19 related deaths in San Bernardino County.The top figure displays the data on a linear scale and the bottom figure displays the data on a logarithmic scale. }
		\label{fig:12}
	\end{center}
\end{figure}

\FloatBarrier

\begin{figure}[!htbp]
	\begin{center}
		\includegraphics[scale=0.7]{SB 3}
		\vspace{1mm}
		\includegraphics[scale=0.7]{SB 4}
		\caption{The top figure shows a scatterplot for cumulative lab-confirmed Covid-19 cases in San Bernardino County versus cumulative Covid-19 related deaths in San Bernardino County. The bottom figure shows the correlation between cumulative lab-confirmed Covid-19 cases and Covid-19 related deaths in San Bernardino County versus lag time, where the highest correlation between the two data sets is at day zero. (i.e. the datastets display the highest correlation when there is no lag time applied) }
		\label{fig:13}
	\end{center}
\end{figure}

\FloatBarrier

\begin{figure}[!htbp]
	\begin{center}
		\includegraphics[scale=0.67]{SB 5}
		\vspace{1mm}
		\includegraphics[scale=0.67]{SB 5.5 }
		\caption{ Daily lab-confirmed Covid-19 positive cases plotted alongside individuals currently admitted to the hospital (including lab-cofirmed Covid-19 positive individuals and symptomatic, Covid-19 suspected individuals who are awaiting the return of Covid-19 lab test results) and individuals currently in the ICU (including lab-cofirmed Covid-19 positive individuals and symptomatic, Covid-19 suspected individuals who are awaiting the return of Covid-19 lab test results). The top figure displays the data on a linear scale and the bottom figure displays the data on a logarithmic scale. Both figures display datasets representing San Bernardino County, California.}
		\label{fig:14}
	\end{center}
\end{figure}

\FloatBarrier




\section{San Francisco County, California}


\begin{figure}[!htbp]
	\begin{center}
		\includegraphics[scale=0.6]{SF 1}
		\vspace{1mm}
		\includegraphics[scale=0.6]{SF 1.5 }
		\caption{ Daily lab-confirmed positive Covid-19 cases in San Francisco County and Daily Covid-19 related deaths in San Francisco County.The top figure displays the data on a linear scale and the bottom figure displays the data on a logarithmic scale. }
		\label{fig:15}
	\end{center}
\end{figure}

\FloatBarrier

\begin{figure}[!htbp]
	\begin{center}
		\includegraphics[scale=0.6]{SF 2}
		\vspace{1mm}
		\includegraphics[scale=0.6]{SF 2.5 }
		\caption{Cumulative lab-confirmed positive Covid-19 cases in San Francisco County and cumulative Covid-19 related deaths in San Francisco County.The top figure displays the data on a linear scale and the bottom figure displays the data on a logarithmic scale. }
		\label{fig:16}
	\end{center}
\end{figure}

\FloatBarrier

\begin{figure}[!htbp]
	\begin{center}
		\includegraphics[scale=0.55]{SF 3}
		\vspace{1mm}
		\includegraphics[scale=0.55]{SF 4}
		\caption{ The top figure shows a scatterplot for cumulative lab-confirmed Covid-19 cases in San Francisco County versus cumulative Covid-19 related deaths in San Francisco County. The bottom figure shows the correlation between cumulative lab-confirmed Covid-19 cases and Covid-19 related deaths in San Francisco County versus lag time, where the highest correlation between the two data sets is at day zero. (i.e. the datastets display the highest correlation when there is no lag time applied)}
		\label{fig:17}
	\end{center}
\end{figure}

\FloatBarrier


\begin{figure}[!htbp]
	\begin{center}
		\includegraphics[scale=0.53]{SF 5}
		\vspace{1mm}
		\includegraphics[scale=0.53]{SF 5.5}
		\caption{ Daily lab-confirmed Covid-19 positive cases plotted alongside individuals currently admitted to the hospital (including lab-cofirmed Covid-19 positive individuals and symptomatic, Covid-19 suspected individuals who are awaiting the return of Covid-19 lab test results) and individuals currently in the ICU (including lab-cofirmed Covid-19 positive individuals and symptomatic, Covid-19 suspected individuals who are awaiting the return of Covid-19 lab test results). The top figure displays the data on a linear scale and the bottom figure displays the data on a logarithmic scale. Both figures display datasets representing San Francisco County, California}
		\label{fig:18}
	\end{center}
\end{figure}

\FloatBarrier




\end{document}
